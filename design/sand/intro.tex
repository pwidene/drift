\section{Introduction and Background}

This report discusses the background, design, and implementation of a software service, Drift, designed to
support \emph{data fusion}.  We use the term to mean the ability to treat distinct, possibly
heterogeneous, pieces of data as a single entity, for the purposes of identification, manipulation, and
analysis.  We address a universe of structured and semi-structured data, from scientific experimental and
simulation results and inputs about which almost all possible metadata is known to word-processor and
spreadsheet documents whose provenance is unclear.  

Indeed, this disparity in metadata availability in large part motivates this project.  An abstract data
type (ADT) is a idiom directly supported by object-oriented programming languages, in which developers
are encouraged to map the intellectual process of data-driven design from the top (i.e. the ``real
world'') down, as opposed to trying to think in detail about varied collections of primitive data types.
Drift represents an exploration of applying this principle at a higher level.  Fast networks, cheap
near-line storage, and the emergence of ``big data'' problems have led engineers and
scientists to develop data-centric design and problem-solving processes.  These processes incorporate many different kinds
of data produced by different methods, with varying degrees of attached metadata, and with lifetimes
exceeding typical project durations or even employee tenures.  

Drift explores the possibility of forming abstractions around data that has no previous explicit
interrelationships.  We see much of the potential usefulness of a service such as Drift in assisting
users to make explicit relationships between data that were previously part of shared understanding, 
cultural assumptions, or hand-me-down knowledge among developers, or artificially separated by
administrative or other boundaries.  Making these relationships 
explicit and available has the potential to reduce misconceptions about how data in organizations is
used.  

Although making it easier to define such abstractions, to fuse data, is a key goal of Drift, it is also
an enabling step for data analysis.  Assigning names to fused data objects is one way to reference them,
but applying analytic techniques to that naming process promises interesting benefits.  In particular,
using components of the fused data itself to drive indexing schemes that support automated learning and
categorization can provide important leverage for queries and data mining-style operations.  In this
Drift is designed to support making fused data objects into ``first-class'' entities for the purposes of
data mining and analysis, to be operated on using analytic methods in the same way as primitive data
types are in current usage.


\subsection{Background and approach}

The database research community was among the first to address problems of data fusion, exploring
approaches such as revised relational algebras and SQL optimization\cite{bleiholder09:_data}. Large-scale availability of
distributed system middleware later drove investigations into federated database and query
services\cite{busse99:_feder_infor_system,sheth90:_feder} in 3-tier enterprise application architectures,
and this flavor of data fusion has persisted through an evolution to service-oriented
architectures\cite{papazoglou03:_servic}.

Drift is intended to improve the availability and utility of federated query-style data services for
researchers in large-data environments comprising multiple heterogeneous sources by making it easier for
them to construct views on distributed data that are meaningful to them. Drift embodies a proposition
that significant expressive power can be obtained by connecting application-specific views to fused and
discrete data sources (flexible multi-indexing), and that tools and software systems that directly
support this will prove useful. Drift's design further assumes that, given such tools and systems,
facilities that provide push-based notification of changes in those views and that can be integrated
directly into applications and data (data proactivity) will also prove useful.

Our method of defining and advertising views on distributed data is more flexible than traditional
federated query approaches in that it encourages a “bottom-up” approach where multiple, perhaps
inconsistent views on the same data are possible. Rather than finding ways to limit the proliferation of
these views, Drift makes them reusable and extensible. Drift also aims to reintroduce distributed query
capabilities for data other than semantically organized text, as much recent research has emphasized
federated text search\cite{sparql}. In contrast to existing data mining/analysis tools such as
Splunk\cite{splunk}, Drift provides programmatic solutions compatible with extremely high-throughput
streaming data environments where indexing, analysis and shaping of metadata and data must be performed
\emph{in situ}, and where advanced analytic indexing capabilities can be introduced directly by scientists and
researchers.


By implementing flexible multi-indexing, Drift enables users to define their own customized views for the
metadata they are concerned with, and directly embed these structures into a metadata service. While
defining indexing structures is a fairly straightforward (and obviously useful) operation for commonly
used hierarchical naming structures such as tries, we believe that more novel results will come from the
use of analytic data structures for indexing. 

For example, as high-dimensional, “big-data” scenarios become commonplace, and in a dynamic metadata
environment such as that we envision, kd-trees\cite{bentley80:_multid} whose dimensional components can
be customized by end-users have the potential to provide extremely granular data selection while still
maintaining efficient indexing. Another example is the use of a kd-tree to implement a nearest-neighbor
search, which can for example complement the automatic classification tasks in the Cyber domain. Of
particular interest are scenarios where metadata objects (potentially comprising multiple objects
spanning multiple backing data stores) can be indexed by more than one user-defined index structure (for
example, consider a feature set generated by computational simulation, indexed both hierarchically by a
trie and by an N- dimensional hyperplane through the feature set).

Users wishing to construct a new fused data product through the use of multiple indexes may wish to know
if related data products are already available so that those products might be reused or extended. Drift
provides a publish/subscribe-like model where entities interested in changes in the data represented in a
service, or changes in the structure of the service itself (addition of new indexes, for example), can be
proactively notified rather than having to interrogate the service.  This style of state management
provides an asynchronous, eventually-consistent global picture of the available views defined by all
users, which is both necessary to enable discovery and encourage reuse, and more flexible and scalable
than federated query approaches enforcing stricter consistency semantics. We also expect proactivity to
provide direct benefits to users, allowing coupling between computational or sensor-based
results. Consider for example the advertisement of a fused data product that must await a cooperating
computation to be completed, or the establishment of a condition driven by asynchronous external
phenomena (such as a malicious network attack detected by a stateful firewall). Proactive interfaces can
also complement the trigger- or change-based mechanisms already available in the backing stores Drift
uses.

Drift uses flexible and scalable schemaless data stores to store and make available
heterogeneously-structured and unstructured data. These tools have been studied in detail at Sandia and
elsewhere, and Drift does not innovate in their design or construction. There are several flavors of
these stores, each representing a particular choice of trade-offs: key-value (Amazon
Dynamo\cite{Hastorun07dynamo:amazons} and descendants), column-oriented (e.g. Google
BigTable\cite{Chang06bigtable:a}, Apache Cassandra), document-oriented (e.g. Couch, Mongo), and
graph/RDF/adjacency (e.g. Neo4j, Allegro). Drift uses this diversity of data representation to more
effectively cope with the variety of data and metadata we expect to encounter.

More interestingly, Drift extends traditional federated-query-style data management by providing the
ability to construct logical views over physical data residing in multiple such stores, leveraging the
unique benefits of each. For example, the best method for representing relationships between documents
might be a graph database, but it could prove useful to associate that relationship with sets of
experimental results residing in a column-store. Lastly, being able to incorporate tools like HBase
(thereby coupling map-reduce computations) enables a further degree of richness in data view definition.

\subsection{Use Cases}

Issues raised in the following use cases motivate our work:

\begin{enumerate}
\item Long-running engineering efforts accumulate vast stores of different but equally important data
  over time as artifacts of design, testing, and production.  Design documents include reports,
  schematics, spreadsheets, emails, and other notes; these are typically produced by and manipulated with
  commercial office productivity software such as Microsoft Office.  Testing data adds to this large
  amounts of numerical results, test descriptions, and parameter sets.  Production data adds another type
  of data store to the problem, as database management software is frequently used to maintain inventory
  information.  When problems are detected during testing or production use, answering the questions that
  lead to root causes requires a holistic look back at a large and interrelated data space: What testing
  regime was used for the widget in question?  was its design valid?  how many of these widgets are in
  production?  Combining different heterogeneous data in the right ways can quickly illuminate these
  issues.

\item Certain corporate data must be mined for features which are relevant to ongoing operations.  For
  instance, streaming telemetry collection from internal cloud deployments can pose issues deriving from
  all four of the ``V's'' of big data: volume, velocity, variety, and veracity.  This telemetry can
  include system health and performance monitoring as well as application data streams.  Fusing these
  data streams to support enterprise decision-making is an important capability (consider, for example,
  an internal cloud-based ``honeypot'' used to isolate, diagnose, and respond to cybersecurity attacks in
  near-real-time).
\end{enumerate}



%%% Local Variables: 
%%% mode: latex
%%% TeX-master: "paper"
%%% End: 

