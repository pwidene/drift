
\section{Conclusion}

Our exploration of data fusion issues with Drift has been illuminating in several respects.  

\begin{itemize}

\item While the Drift software itself has enjoyed limited penetration in the Sandia software development
  community, the ideas behind it still resonate with developers and designers working with diverse data.
  Sandia's Cyber community appears satisfied with the Splunk toolset.  However, Splunk's licensing costs are
  burdensome enough that other domain researchers are starting developments projects, based on
  open-source software, with capability sets similar to both Splunk and Drift.  While the Drift
  implementation may not be ideal infrastructure for all such efforts, there is clearly value in
  establishing this type of software/services layer for use across the Laboratories.

\item A corollary to the previous point is that Laboratory funding for data analytics software should be
  carefully rationalized.  At the beginning of this project, the intended feature set for Drift subsumed
  the capabilities of the Splunk installed base.  12 months later, that situation was dramatically
  reversed, owing entirely to a growing community of Sandians focused on Splunk extension and integration
  development.  Unfortunate timing/coscheduling has been the cause of much unrealized potential in software
  development in large organizations, and part of the research agenda is, and properly, to support
  simultaneous investigation into broad areas (such as software to support data science) in order to gain
  multiple perspectives on large and stubborn issues.  Notwithstanding this, and build-vs-buy
  considerations aside, we believe decisions by the Laboratories to invest large amounts of
  funding (for licensing and development effort) into products like Splunk, creating lock-in engagements
  with software that is itself built from open-source components, should be considered carefully.

\item An ongoing priority for the Laboratories, properly within the purview of the Data Science Research
  Challenge, is to establish procedures for gaining research access to operational data.  In multiple
  instances during this project, goals of the project were significantly impeded by the decision of a
  single individual who happened to be the owner of a particular set of operational data.  No indictment
  of any kind of those individuals is intended here; they are responsible for important components of
  Sandia's operational infrastructure and have acted in good faith with respect to their mission(s).
  However, processes for gaining access to operational data for research purposes should be formalized so
  that they are well-understood and can be applied in an objective manner.  

\end{itemize}

We will be licensing the Drift code as open-source by the end of FY14.  A test suite (including a
document corpus drawn from \cite{govdocs1}) is also planned for inclusion.  Our intention is to explore
future collaborations with university partners that have common interest in the research areas explored
by Drift.  We hope to obtain future funding to continue developing these ideas from the LDRD program,
DOE, or external agencies such as the National Science Foundation.  




%%% Local Variables: 
%%% mode: latex
%%% TeX-master: "paper"
%%% End: 

